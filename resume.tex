% LaTeX resume using res.cls
\documentclass{res} 
%\usepackage{helvetica} % uses helvetica postscript font (download helvetica.sty)
%\usepackage{newcent}   % uses new century schoolbook postscript font 
%\setlength{\parindent}{-10pt} % Default is 15pt.

\addtolength{\oddsidemargin}{-.2in}
\addtolength{\evensidemargin}{-.2in}

\begin{document}

\name{Michael Ouimet, Ph.D.} 
% \address used twice to have two lines of address
%\address{}
%\address{}

 
\begin{resume}

%\section{CONTACT}% INFORMATION} 
%\vspace{-7ex}
%\hspace{-13mm}
%\begin{minipage}{.45\linewidth} 
%{\footnotesize 
%\vspace{5ex}
%mike.ouimet@gmail.com 
%
%San Diego, CA 92109}
%}
%\end{minipage} 
%\hspace*{75 mm} 
%\begin{minipage}{.45\linewidth} 
%\vspace{5ex}
% {\footnotesize
% phone number here
%
%}
% \end{minipage}

 \vspace*{-2ex}
\hspace{-8.5ex}\rule{16.5cm}{0.4pt}
 \vspace*{-3ex}
\section{SUMMARY}
\vspace{1ex}
Engineer with interdisciplinary research experience in planning, control and estimation theory, dynamical systems, and machine learning focusing on robotic applications.  Broad range of mathematical modeling, analysis skills, and programming experience (Python, ROS, and some C) with supervisory roles. Strong written and oral communication skills.

 \vspace*{-4ex}
\hspace{-8.5ex}\rule{16.5cm}{0.4pt}
 \vspace*{-3ex}
\section{EXPERIENCE}
{\sl \bf Lead Engineer at SPAWAR Systems Center Pacific} \hfill July 2017 - present

\vspace*{-2ex}
 Act as both an individual software development contributor as well as autonomy lead and mentor for several projects involving multi-agent autonomy.  Personal contributions include: 
\begin{itemize}
\item Simulating a wave-aware path planner for an unmanned surface vehicle to safely maneuver through rough sea state
\item Developing the controls, estimation, and computer vision algorithms in ROS to allow a Commercial, Off-the-Shelf drone to localize and robustly land on a desired platform using a downward-facing camera.

\end{itemize}
\vspace*{-1ex}
{\sl \bf Engineer at SPAWAR Systems Center Pacific} \hfill July 2015 - July 2017 
\begin{itemize}
\item Principal Investigator (PI) on a proposal funded by the Office of Naval Research (ONR) to co-advise a Ph.D. student at UC San Diego in topics of cooperative estimation and control of unmanned vehicles, especially focused on human-swarm teaming (FY17-FY19)
\item Co-PI and Algorithm Lead on an internally funded research project on Human-Autonomy Teaming. Implemented a number of Artificial Intelligence motion planning algorithms (A*, MDPs, model-predictive control, supervised learning, reinforcement learning) in Python and evaluated their applicability on solving successively more challenging (and therefore realistic) Navy scenarios.
\item Aided in the design of a PID control law for an unmanned military vehicle.  Developed a training document for applying multisensor data fusion techniques (SLAM, Kalman Filter and extensions, Particle Filter, data association) to unmanned military vehicles.
\end{itemize}
\vspace*{-1ex}
{\sl \bf Postdoctoral researcher at UC San Diego} \hfill Apr 2014 - Jun 2015 
\begin{itemize}
\item Developed mathematical model for information throughput of team of aerial, mobile, internet-providing vehicles and designed novel motion control algorithm to deploy over Earth's atmosphere, providing optimal service subject to motion constraints.
%
\item Created event-triggering strategy for cooperative localization algorithm to provide robots' ability to estimate own location cooperatively while minimizing communication costs. Compared novel algorithm to standard baseline algorithms to investigate the tradeoff between communication cost and system performance.
\item Supervised undergraduate and graduate researchers in individual projects towards developing capabilities of lab's new multi-robot testbed, helped determine and direct the long-term focus of the testbed.  Team successfully demonstrated several algorithms including localization, SLAM, multi-agent deployment, and cyclic pursuit. 
\end{itemize}
\vspace*{-1ex}
{\sl  \bf Graduate student at UC San Diego} \hfill Sep 2009 - Apr 2014 
\begin{itemize}
\item Worked on interdisciplinary team with Scripps Institute of Oceanography research engineers and oceanographers to develop algorithms for underwater buoyancy-controlled drifters. Designed, simulated, and validated cooperative algorithms to estimate parameters of ocean internal waves off coast of San Diego.  Algorithm estimated the desired parameters from real data and results are being transferred to the oceanographic community.
\item Developed and simulated novel motion control algorithm where ocean drifters utilize their knowledge of the ocean wave flowfield to maneuver in desired direction by moving vertically in the water column, harnessing varying water speeds. Received honorable mention for presentation of this work at the UCSD Engineering Research Expo 2014.
\item Developed and simulated cooperative algorithm to optimally deploy team of unreliable mobile robotic sensors across an environment. Algorithm has applications in area surveillance and monitoring of environments where communication is challenging, such as underwater. 
\end{itemize}
\vspace*{-2ex}
 {\sl \bf Engineering internship at Otis Elevator Company} \hfill Summer 2006, 2007 
\begin{itemize}
\item Measured elevator noise and developed a script to analyze its frequency spectrum for the purpose of fault detection. 
\item Documented, re-implemented, and added functionality to a script of an elevator belt wear model.  This piece of software was easier for anyone to edit, well documented, and had one tenth the runtime.
\end{itemize}
%
 \vspace*{-3ex}
\hspace{-8.5ex}\rule{16.5cm}{0.4pt}
 \vspace*{-3ex}
\section{ADDITIONAL INFORMATION}
\begin{itemize}
\item Audited Micro-MBA course for academics through the UCSD Rady School of Management
\item Led lab outreach tours for High School and University groups to inspire students to join STEM fields and consider careers in robotics/research
%\item {\bf Interests: }Reading, surfing, rock climbing, photography, travel
\end{itemize}


 \vspace*{-3ex}
\hspace{-8.5ex}\rule{16.5cm}{0.4pt}
 \vspace*{-3ex}
\section{EDUCATION} 
\textbf{University of California, San Diego}, California, USA  \\
\vspace{0ex}
\quad Ph.D., Mechanical and Aerospace Engineering (Advisor: Jorge Cort\'es), Apr. 2014 \\
\vspace{0ex}
\qquad -Thesis title: Distributed Cooperation for Robust Estimation\\
\vspace{0ex}
%\qquad -Advisor: Jorge Cort\'es\\
%
%\vspace*{1ex}
\quad M.S., Mechanical and Aerospace Engineering, Nov. 2010\\
%
\vspace*{0ex}
\textbf{University of Notre Dame}, Indiana, USA \\
%
\vspace*{0ex}
\quad B.S., Mechanical Engineering, Jun. 2009
\vspace*{-3ex}
% %\section{QUALIFICATIONS}
% %\begin{description}
% %\item[Dynamic systems and control courses:] Linear and nonlinear
% %  systems theory, Linear and nonlinear control, Adaptive Control,
% %  Hybrid systems, Optimal control, Optimal estimation, Cooperative control, Stochastic
% %  processes, Robot motion planning 
% %\item[Algorithm design:]  Strong research experience in the design, simulation, and rigorous analysis of algorithms for correctness, time complexity, and robustness performance. 
% %\item[Programming experience:] Exceptional at MATLAB, proficient in
% %  Mathematica, familiar with Python
% %\end{description}
% %\vspace*{-3ex}
% %\section{RESEARCH PROJECT INVOLVEMENT}
% %%\\
% %{\it Implementing distributed algorithms on a multi-agent robotic platform} - Jan. 2014 - Present \\
% %\vspace{-2ex}
% %\begin{itemize}
% %\item Currently helping to manage and organize Dr. Cortes' and Dr. Martinez's joint robotic lab, including leading weekly meetings and supervising all undergraduate/Masters students (currently 4 direct reports)
% %\item The team is working towards implementing distributed motion coordination and estimation algorithms(deployment, cyclic pursuit, leader following, SLAM) on a team of 10 Turtlebots
% %\end{itemize}
% %\vspace{-2ex}
% %{\it Distributed Ocean Monitoring via Integrated Data Analysis of Coordinated Buoyancy Drogues}, NSF - Division of Ocean Sciences (OCE-0941692) - Sep. 2010 - Mar. 2014  \\
% %\vspace{-2ex}
% %\begin{itemize}
% %\item Collaborative research project between UCSD control theorists and Scripps Institute of Oceanography ocean scientists to estimate ocean phenomena
% %\item  Developed novel algorithms to be run on data collected by Scripps scientists' mobile robotic sensors
% %\item Tested their efficacy by applying the algorithms to real collected data
% %\end{itemize}
% %\vspace{-4ex}

% % \vspace*{-4.5ex}
% \hspace{-8.5ex}\rule{16.5cm}{0.4pt}
%  \vspace*{-3ex}
% \section{JOURNAL PUBLICATIONS} 
% \vspace{0ex}
% M. Ouimet and J. Cort\'es. Robust coordinated rendezvous of depth-actuated drifter in ocean internal waves. Automatica, vol. 69, pp. 265-274, 2016\\

% \vspace*{-4ex}

% M. Ouimet and J. Cort\'es. Robust, distributed estimation of internal wave parameters via inter-drogue measurements. IEEE Transactions on Control Systems Technology, vol. 22, no. 3, pp.~980-994,~2014\\
% \vspace*{-4ex}

% M. Ouimet and J. Cort\'es. Collective estimation of ocean nonlinear internal waves using robotic underwater drifters. IEEE Access, vol. 1, pp. 418-427, 2013. \\

% \vspace*{-4ex}
% M. Ouimet and J. Cort\'es. Hedonic coalition formation for optimal deployment.
% Automatica, vol. 49, no. 11, 3234-3245, 2013 \\
% \vspace{-6ex}

% \hspace{-8.5ex}\rule{16.5cm}{0.4pt}
%  \vspace*{-3ex}
% \section{CONFERENCE PROCEEDINGS}
% \vspace{0ex}
% M. Ouimet, N. Ahmed, and Sonia Mart\'inez. Event-based cooperative localization using implicit and explicit measurements. Conference on Multisensor Fusion and Integration for Intelligent Systems, San Diego, CA, 2015

% M. Ouimet, J. Cort\'es, and Sonia Mart\'inez. Global network integrity using altitude-actuating balloons in the stratosphere.  Conference on Decision and Control, Osaka, Japan, 2015 \\

% \vspace*{-4ex}

% M. Ouimet and J. Cort\'es. Coordinated rendezvous of underwater drogues in ocean internal waves. Conference on Decision and Control, Los Angeles, CA, 2014\\

% \vspace*{-4ex}
% M. Ouimet and J. Cort\'es. Robust estimation and aggregation of ocean internal wave parameters using Lagrangian drifters. American Control Conference, Portland, OR, 2014\\

% \vspace*{-4ex}
%  M. Ouimet and J. Cort\'es.  Distributed estimation of internal wave parameters via inter-drogue distances.
% Conference on Decision and Control, Maui, HI, 2012 \\

% \vspace*{-4ex} M. Ouimet and J. Cort\'es. Coalition formation and
% motion coordination for optimal deployment.  Conference on Decision and Control,
% Orlando, FL, 2011
% %\section{RESEARCH EXPERIENCE}
% % {\sl Graduate Student Researcher at UCSD} \hfill Summer 2009-Present \\
% %blahblah
% \vspace*{-2ex}

% %\section{LEADERSHIP}
% %{\sl Mentored undergraduate students} \hfill Summer 2012, Fall 2012, Spring 2013, Winter 2014, Spring 2014 \\
% %%{\sl Adam Durbin} \hfill Fall 2012 \\
% %%{\sl Ethan Allen} \hfill Summer 2012 \\
% % -designed quarter-length research projects on the topic of `Computing and employing Voronoi Diagrams for applications in mobile sensor networks' and advised the students through mentorship
% %\\
% %-currently supervising three undergraduate researchers working on two robot localization projects, whose goals are to implement two different methods for our robotic network to gain position information (GPS via a webcam's vision and Simultaneous Localization And Mapping (SLAM))
% %
% % {\sl Teaching Assistant} \hfill Fall 2009, Fall 2011 \\
% %                MAE 140 - Linear Circuits, UCSD  \\
% %-for a 200 student class, I ran small group office hours, led large discussion sections, and graded homework and exams

% \vspace*{-2ex}
% \hspace{-8.5ex}\rule{16.5cm}{0.4pt}
%  \vspace*{-3ex}
% \section{AWARDS AND HONORS} 
% \vspace{0ex}
% Invited speaker to UC Irvine Mechanical Engineering department, April 2016 \\
% \vspace*{-4.5ex}

% Nominated for Outstanding Contribution award for Transactions on Control Systems Technology \\
% \vspace*{-4.5ex}

% Honorable mention at the UCSD Engineering Research Expo, 2014

% \vspace*{-3ex}

%  \vspace*{-2ex}
% \hspace{-8.5ex}\rule{16.5cm}{0.4pt}
%  \vspace*{-3ex}
%\section{TALKS AND POSTERS DELIVERED} 
%\vspace{0ex}
%Conference on Multisensor Fusion and Integration for Intelligent Systems, San Diego, CA, 2015\\
%\vspace*{-4.5ex}
%
%IEEE Conference on Decision and Control 2014, Los Angeles, CA, 2014 \\
%\vspace*{-4.5ex}
%
%American Control Conference, 2014, Portland, OR, 2014\\
%\vspace*{-4.5ex}
%
%Engineering Research Expo, University of California, San Diego, CA, 2014 - Honorable mention \\
%\vspace*{-4.5ex}
%
%26th Southern California Nonlinear Controls Workshop, University of California, Santa Barbara, 2014\\
%\vspace*{-4.5ex}
%
%SIAM Conference for Control and its Applications, San Diego, CA, 2013, invited speaker for minisymposium on `Marine Robotic Controls'\\
%\vspace*{-4.5ex}
%
%Engineering Research Expo, University of California, San Diego, CA, 2013 \\
%\vspace*{-4.5ex}
%
%IEEE Conference on Decision and Control 2012, Maui, HI, 2012 \\
%\vspace{-4.5ex}
%
%22nd Southern California Nonlinear Controls Workshop, University of Southern California, Los Angeles, CA, 2012 \\
%\vspace*{-4.5ex}
%
%IEEE Conference on Decision and Control, Orlando, FL, USA, 2011\\
%\vspace*{-4.5ex}
%
 %National Control Engineering Students Workshop, University of Maryland,College Park, MD, 2011 \\
%\vspace*{-4.5ex}

%Engineering Research Expo, University of California, San Diego CA, 2011 

 %\vspace*{-3ex}
%\hspace{-8.5ex}\rule{16.5cm}{0.4pt}
% \vspace*{-3ex}
%\vspace*{-5ex}
%%
%%Weekly research group paper presentations, UCSD, 2009-present
%\vspace*{-4.5ex}
%\section{RESEARCH INTERESTS}       
%Dynamical systems and control; Mobile sensor networks; Distributed
%coordination algorithms; Cooperative control; Robotics; Underwater vehicles; Parameter estimation; Game theory; Ad hoc wireless networks; distributed state estimation
%\vspace*{-2.5ex}
%\section{ADDITIONAL INTERESTS}
%surfing, rock climbing, photography, travel, reading


\end{resume}
\end{document}







